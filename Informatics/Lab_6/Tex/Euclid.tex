\documentclass[a4paper, 12 pt]{extarticle}
\usepackage{fancyhdr}
\usepackage[russian]{babel}
\usepackage[utf8]{inputenc}
\usepackage[textwidth = 18 cm, left = 1 cm,
top = 2.5 cm, headsep = 1 cm]{geometry}
\usepackage{lipsum}
\usepackage{graphicx, lettrine}
\usepackage{eufrak}
\usepackage{tikz}
\usetikzlibrary{angles,quotes}

\begin{document}
\pagestyle{fancy}
\renewcommand{\headrulewidth}{0pt}
\fancyhf{}
\fancyhead[R]{\MakeUppercase{книга iii предл. iv. теорема \qquad}}
\setcounter{page}{96}
\fancyhead[C]{$\mathfrak {\thepage}$}
\begin{minipage}{0.45 \textwidth}
\begin{tikzpicture}[scale = 1.7]
\draw[line width = 3pt, violet](0,0) circle (2cm);
\coordinate (a) at (-0.08,-1);
\coordinate (b) at (0.04,1);
\coordinate (c) at (1,-0.8);
\draw pic[draw,fill=blue,angle radius=0.7 cm] {angle=c--a--b};
\coordinate (m) at (-0.08,-1);
\coordinate (n) at (2,-0.6);
\coordinate (k) at (3,-2);
\draw pic[draw,fill=yellow,angle radius=0.7 cm] {angle=k--m--n};
\draw[line width = 2.5 pt, brown] (220:2) -- (-20:2);
\node[below left] at (220:2) {\tiny $B$};
\node[right] at (-20:2) {\tiny $D$};
\draw[line width = 2.5 pt] (190:2) -- (310:2);
\node[left] at (190:2) {\tiny $A$};
\node[below right] at (310:2) {\tiny $C$};
\draw[densely dashed, line width = 2.5 pt] (0,0) -- (-95:1);
\node[above] at (0,0) {\tiny $F$};
\node[below] at (-95:1.1) {\tiny $E$};
\end{tikzpicture}
\vspace{3.5 in}
\end{minipage}
\begin{minipage}{0.55 \textwidth}
\lettrine[image=true, lines = 4]{letter.jpg}{}{ сли} {\it в круге две прямые, не проходящие через центр, пересекаются,они не делят друг друга пополам.}

\vspace{0.8 cm}

\setlength{\parindent}{20pt}
Если одна из прямых проходит через центр, очевидно, она ее не может рассекать пополам другая прямая, не проходящая через центр.

Но если ни одна из прямых
\begin{tikzpicture}
\draw[line width = 2.5 pt] (0,0) -- (1, 0);
\node[above] at (0,0) {\tiny $A$};
\node[above] at (1, 0) {\tiny $C$};
\end{tikzpicture} или
\begin{tikzpicture}
\draw[line width = 2.5 pt, brown] (0,0) -- (1, 0);
\node[above] at (0,0) {\tiny $B$};
\node[above] at (1, 0) {\tiny $D$};
\end{tikzpicture}
не проходит через центр, проведем
\begin{tikzpicture}
\draw[densely dashed, line width = 2.5 pt] (0,0) -- (1, 0);
\node[above] at (0,0) {\tiny $E$};
\node[above] at (1, 0) {\tiny $F$};
\end{tikzpicture} 
из центра к точке их пересечения.

\begin{center}
\begin{minipage}{7.4 cm}
\centering
Если
\begin{tikzpicture}
\draw[line width = 2.5 pt] (0,0) -- (1, 0);
\node[above] at (0,0) {\tiny $A$};
\node[above] at (1, 0) {\tiny $C$};
\end{tikzpicture}
делится пополам,
\begin{tikzpicture}
\draw[densely dashed, line width = 2.5 pt] (0,0) -- (1, 0);
\node[above] at (0,0) {\tiny $E$};
\node[above] at (1, 0) {\tiny $F$};
\end{tikzpicture}
$\perp$ ей (пр. III$\mathfrak {.3}$)

\begin{tikzpicture}
\fill (0pt, 20pt) circle (1pt);
\fill (6pt, 20pt) circle (1pt);
\fill (3pt, 25pt) circle (1pt);
\fill (3pt, 0pt) circle (0pt)
\end{tikzpicture}
\begin{tikzpicture}
\coordinate (a) at (0, 0);
\coordinate (b) at (80:0.3);
\coordinate (c) at (10:0.3);
\draw pic[draw,fill=blue,angle radius=0.7 cm] {angle=c--a--b};
\coordinate (m) at (0,0);
\coordinate (n) at (10:0.3);
\coordinate (k) at (-15:0.3);
\draw pic[draw,fill=yellow,angle radius=0.7 cm] {angle=k--m--n};
\node[below left] at (0,0) {\tiny $E$};
\node[below] at (-15:0.7) {\tiny $C$};
\node[above] at (80:0.7) {\tiny $F$};
{\hspace{1 cm}\LARGE $=$}
\end{tikzpicture} 
\hspace{0.75cm}
\begin{tikzpicture}
\draw[ultra thick] (-0.75,0.5) -- (0, 0.5) -- (0,1.25);
\draw[line width = 3pt] (0, 1.25) arc (90:180:0.75);
\fill (3pt, 0pt) circle (0pt)
\end{tikzpicture}

и если \begin{tikzpicture}
\draw[line width = 2.5 pt, brown] (0,0) -- (1, 0);
\node[above] at (0,0) {\tiny $B$};
\node[above] at (1, 0) {\tiny $D$};
\end{tikzpicture}
делится пополам,
\begin{tikzpicture}
\draw[densely dashed, line width = 2.5 pt] (0,0) -- (1, 0);
\node[above] at (0,0) {\tiny $E$};
\node[above] at (1, 0) {\tiny $F$};
\end{tikzpicture}
$\perp$
\begin{tikzpicture}
\draw[line width = 2.5 pt, brown] (0,0) -- (1, 0);
\node[above] at (0,0) {\tiny $B$};
\node[above] at (1, 0) {\tiny $D$};
\end{tikzpicture}
(пр. III$\mathfrak {.3}$)

\begin{tikzpicture}
\fill (0pt, 20pt) circle (1pt);
\fill (6pt, 20pt) circle (1pt);
\fill (3pt, 25pt) circle (1pt);
\fill (3pt, 0pt) circle (0pt)
\end{tikzpicture}
\begin{tikzpicture}
\coordinate (a) at (0, 0);
\coordinate (b) at (80:0.3);
\coordinate (c) at (10:0.3);
\draw pic[draw,fill=blue,angle radius=0.7 cm] {angle=c--a--b};

\node[below left] at (0,0) {\tiny $E$};
\node[below] at (10:0.7) {\tiny $D$};
\node[above] at (80:0.7) {\tiny $F$};
{\hspace{1 cm}\LARGE $=$}
\end{tikzpicture} 
\hspace{0.75cm}
\begin{tikzpicture}
\draw[ultra thick] (-0.75,0.4) -- (0, 0.4) -- (0,1.15);
\draw[line width = 3pt] (0, 1.15) arc (90:180:0.75);
\fill (3pt, 0pt) circle (0pt)
\end{tikzpicture}

\begin{tikzpicture}
\fill (0pt, 20pt) circle (1pt);
\fill (6pt, 20pt) circle (1pt);
\fill (3pt, 25pt) circle (1pt);
\fill (3pt, 0pt) circle (0pt)
\end{tikzpicture}
\begin{tikzpicture}
\coordinate (a) at (0, 0);
\coordinate (b) at (80:0.3);
\coordinate (c) at (10:0.3);
\draw pic[draw,fill=blue,angle radius=0.7 cm] {angle=c--a--b};
\node[below left] at (0,0) {\tiny $E$};
\node[below] at (10:0.7) {\tiny $B$};
\node[above] at (80:0.7) {\tiny $F$};
{\hspace{1 cm}\LARGE $=$}
\end{tikzpicture} 
\hspace{0.5cm}
\begin{tikzpicture}
\coordinate (a) at (0, 0);
\coordinate (b) at (80:0.3);
\coordinate (c) at (10:0.3);
\draw pic[draw,fill=blue,angle radius=0.7 cm] {angle=c--a--b};
\coordinate (m) at (0,0);
\coordinate (n) at (10:0.3);
\coordinate (k) at (-15:0.3);
\draw pic[draw,fill=yellow,angle radius=0.7 cm] {angle=k--m--n};
\node[below left] at (0,0) {\tiny $E$};
\node[below] at (-15:0.7) {\tiny $C$};
\node[above] at (80:0.7) {\tiny $F$};
\end{tikzpicture}

часть равна целому, что невозможно.
\end{minipage}

\begin{tikzpicture}
\fill (0pt, 0pt) circle (1pt);
\fill (6pt, 0pt) circle (1pt);
\fill (3pt, 5pt) circle (1pt);
\end{tikzpicture}
\begin{tikzpicture}
\draw[line width = 2.5 pt] (0,0) -- (1, 0);
\node[above] at (0,0) {\tiny $A$};
\node[above] at (1, 0) {\tiny $C$};
\end{tikzpicture} 
и 
\begin{tikzpicture}
\draw[line width = 2.5 pt, brown] (0,0) -- (1, 0);
\node[above] at (0,0) {\tiny $B$};
\node[above] at (1, 0) {\tiny $D$};
\end{tikzpicture}
не делят друг друга пополам.
\end{center}
\begin{flushright}
\vspace{-1.5 ex}
ч.т.д.
\end{flushright}
\end{minipage}
\end{document}